%%%%%%%% ICML 2024 EXAMPLE LATEX SUBMISSION FILE %%%%%%%%%%%%%%%%%

\documentclass{article}

% Recommended, but optional, packages for figures and better typesetting:
\usepackage{microtype}
\usepackage{graphicx}
\usepackage{subfigure}
\usepackage{booktabs} % for professional tables

% hyperref makes hyperlinks in the resulting PDF.
% If your build breaks (sometimes temporarily if a hyperlink spans a page)
% please comment out the following usepackage line and replace
% \usepackage{icml2024} with \usepackage[nohyperref]{icml2024} above.
\usepackage{hyperref}


% Attempt to make hyperref and algorithmic work together better:
\newcommand{\theHalgorithm}{\arabic{algorithm}}

% Use the following line for the initial blind version submitted for review:
% \usepackage{icml2024}

% If accepted, instead use the following line for the camera-ready submission:
\usepackage[accepted]{icml2024}

% For theorems and such
\usepackage{amsmath}
\usepackage{amssymb}
\usepackage{mathtools}
\usepackage{amsthm}

% if you use cleveref..
\usepackage[capitalize,noabbrev]{cleveref}

%%%%%%%%%%%%%%%%%%%%%%%%%%%%%%%%
% THEOREMS
%%%%%%%%%%%%%%%%%%%%%%%%%%%%%%%%
\theoremstyle{plain}
\newtheorem{theorem}{Theorem}[section]
\newtheorem{proposition}[theorem]{Proposition}
\newtheorem{lemma}[theorem]{Lemma}
\newtheorem{corollary}[theorem]{Corollary}
\theoremstyle{definition}
\newtheorem{definition}[theorem]{Definition}
\newtheorem{assumption}[theorem]{Assumption}
\theoremstyle{remark}
\newtheorem{remark}[theorem]{Remark}

% Todonotes is useful during development; simply uncomment the next line
%    and comment out the line below the next line to turn off comments
%\usepackage[disable,textsize=tiny]{todonotes}
\usepackage[textsize=tiny]{todonotes}


% The \icmltitle you define below is probably too long as a header.
% Therefore, a short form for the running title is supplied here:
\icmltitlerunning{ETH Zürich - Deep Learning class 2024-2025}

\begin{document}

\twocolumn[
\icmltitle{ETH Zürich - Deep Learning class 2024-2025}

% It is OKAY to include author information, even for blind
% submissions: the style file will automatically remove it for you
% unless you've provided the [accepted] option to the icml2024
% package.

% List of affiliations: The first argument should be a (short)
% identifier you will use later to specify author affiliations
% Academic affiliations should list Department, University, City, Region, Country
% Industry affiliations should list Company, City, Region, Country

% You can specify symbols, otherwise they are numbered in order.
% Ideally, you should not use this facility. Affiliations will be numbered
% in order of appearance and this is the preferred way.
\icmlsetsymbol{equal}{*}

\begin{icmlauthorlist}
\icmlauthor{Woojin Ban}{}
\icmlauthor{Nino Courtecuisse}{}
\icmlauthor{Tom Nonnenmacher}{}
\icmlauthor{Thomas Zilliox}{}
%\icmlauthor{}{sch}
%\icmlauthor{}{sch}
\end{icmlauthorlist}

%\icmlaffiliation{yyy}{Department of XXX, University of YYY, Location, Country}
%\icmlaffiliation{comp}{Company Name, Location, Country}
%\icmlaffiliation{sch}{School of ZZZ, Institute of WWW, Location, Country}

\icmlcorrespondingauthor{Firstname1 Lastname1}{first1.last1@xxx.edu}
\icmlcorrespondingauthor{Firstname2 Lastname2}{first2.last2@www.uk}

% You may provide any keywords that you
% find helpful for describing your paper; these are used to populate
% the "keywords" metadata in the PDF but will not be shown in the document
\icmlkeywords{Machine Learning, Deep Learning, Continuous Learning, Lifelong Learning, Complexity, Linear Regions}

\vskip 0.3in
]

% this must go after the closing bracket ] following \twocolumn[ ...

% This command actually creates the footnote in the first column
% listing the affiliations and the copyright notice.
% The command takes one argument, which is text to display at the start of the footnote.
% The \icmlEqualContribution command is standard text for equal contribution.
% Remove it (just {}) if you do not need this facility.

%\printAffiliationsAndNotice{}  % leave blank if no need to mention equal contribution
%\printAffiliationsAndNotice{\icmlEqualContribution} % otherwise use the standard text.

\begin{abstract}
    This is the project proposal of our group for the Deep Learning class 2024-2025. We choose to work on continuous learning and our goal is to study the influence of model complexity on catastraphic forgetting.
\end{abstract}



\section{Introdution}
As the causes of atastrophic forgetting\cite{catastrophic_forgetting} are not yet fully understood, we want to measure the impact that specific features can have on catastrophic forgetting. We will focus on the impact of model complexity on forgetting. As model complexity can have many definition, see for example the survey\cite{dl_model_complexity_survey}, we first define which measures we will use and then we detail the experimental procedure.

\section{Procedure}
\subsection{Complexity measure}
To quantify model complexity, two metrics will be used: 
\begin{itemize}
	\item[-] the number of parameters for the expressive complexity,
	\item[-] the number of active neurons or the number of linear regions for effective complexity.
\end{itemize}
The second one is particularly interesting, as according to\cite{hanin2019complexitylinearregionsdeep}, it seems to capture different stages during training.
\subsection{Benchmark}
As a benchmark, we will use a multi-task scenario with commonly used datasets such as splitted MNIST.

\subsection{Model selection and experiment}
As a baseline, a simple model such as MLP will be used with minimum number of parameters to reach a decent accuracy (greater than 95-98 \%). \\
We will benchmark accuracy loss at each step of the multi-task scenario, varying the number of training epochs, model width and depth and repeating the process multiple times.

\subsection{Continuous learning}
We will perform experiments on continuous learning algorithms such as "Efficient Lifelong Learning with A-GEM"\cite{chaudhry2019efficientlifelonglearningagem} or "Learning Without Forgetting"\cite{li2017learningforgetting}. \\
The continuous learning framework that we selected is the open source framework "mammoth"\cite{mammoth}.

\bibliography{example_paper}
\bibliographystyle{icml2024}
\end{document}


% This document was modified from the file originally made available by
% Pat Langley and Andrea Danyluk for ICML-2K. This version was created
% by Iain Murray in 2018, and modified by Alexandre Bouchard in
% 2019 and 2021 and by Csaba Szepesvari, Gang Niu and Sivan Sabato in 2022.
% Modified again in 2024 by Sivan Sabato and Jonathan Scarlett.
% Previous contributors include Dan Roy, Lise Getoor and Tobias
% Scheffer, which was slightly modified from the 2010 version by
% Thorsten Joachims & Johannes Fuernkranz, slightly modified from the
% 2009 version by Kiri Wagstaff and Sam Roweis's 2008 version, which is
% slightly modified from Prasad Tadepalli's 2007 version which is a
% lightly changed version of the previous year's version by Andrew
% Moore, which was in turn edited from those of Kristian Kersting and
% Codrina Lauth. Alex Smola contributed to the algorithmic style files.
